\documentclass[paper=a4,fontsize=10pt,DIV=11,BCOR=3mm,pdftex]{scrartcl}

\usepackage{amsmath,amssymb,amsthm}
\usepackage{lmodern}
\usepackage[utf8]{inputenc}
\usepackage{mathtools}
\usepackage{tikz-cd}
\usepackage{hyperref}

\setlength\parindent{0pt}
\renewcommand{\labelenumi}{(\roman{enumi})}
\linespread{1.0}


\begin{document}
\thispagestyle{empty}
{
\centering
\Large \textbf{Finn Bartsch -- Curriculum Vitae} \par
}
\bigskip \par
\textit{E-Mail:}~\href{mailto:f.bartsch@math.ru.nl}{f.bartsch@math.ru.nl} \\
\textit{Mailing address:}~PO Box 9010, 6500 GL Nijmegen, The Netherlands \\
\textit{Visiting address:}~Office HG03.084, Huygens building, Heyendaalseweg 135, 6525 AJ Nijmegen \\
\par

\subsection*{Education}
PhD Mathematics, expected graduation Fall 2026 \\
Radboud-Universiteit Nijmegen, 2022-2026 \\
Thesis advisor: Ariyan Javanpeykar \\
~\par

M.~Sc.~Mathematics, obtained in September 2022 \\
Johannes Gutenberg-Universität Mainz, 2020-2022 \\
Thesis advisor: Ariyan Javanpeykar \\
Thesis title: \textit{Varieties with many rational points over function fields} \\
~\par

B.~Sc.~Mathematics, obtained in June 2020 \\
Johannes Gutenberg-Universität Mainz, 2017-2020 \\
Thesis advisor: Manuel Blickle \\
Thesis title: \textit{Delta-Ringe}

~\par


\subsection*{Papers}

\href{https://arxiv.org/abs/2502.09414}{\textit{On the finiteness of maps into simple abelian varieties satisfying certain tangency conditions}} \\
Preprint, submitted (2025). \\

\href{https://arxiv.org/abs/2502.09400}{\textit{New examples of geometrically special varieties: K3 surfaces, Enriques surfaces, and algebraic groups}} \\
Preprint, submitted (2025). \\

\href{https://arxiv.org/abs/2412.14931}{\textit{Symmetric products and puncturing Campana-special varieties}} (joint with Ariyan Javanpeykar and Aaron Levin) \\
Preprint, submitted (2024). \\

\href{https://arxiv.org/abs/2410.06643}{\textit{The Weakly Special Conjecture contradicts orbifold Mordell, and thus abc}} (joint with Frédéric Campana, Ariyan Javanpeykar, and Olivier Wittenberg) \\
Preprint, submitted (2024). \\

\href{https://doi.org/10.1016/j.indag.2024.10.005}{\textit{Parshin's method and the geometric Bombieri--Lang conjecture}} (joint with Ariyan Javanpeykar) \\
Indagationes Mathematicae, Jacob Murre special issue, to appear. \\

\href{https://arxiv.org/abs/2310.09065}{\textit{Weakly-special threefolds and non-density of rational points}} (joint with Ariyan Javanpeykar and Erwan Rousseau) \\
Preprint, submitted (2023). \\

\href{https://doi.org/10.1017/S1474748024000094}{\textit{Kobayashi--Ochiai's finiteness theorem for orbifold pairs of general type}} (joint with Ariyan Javanpeykar) \\
Journal of the Institute of Mathematics of Jussieu (2024).

~\par


\subsection*{Teaching}
Teaching assistant, Radboud-Universiteit Nijmegen, 2023- \\
Exercise classes and grading for \textit{Riemann surfaces}, \textit{Galois Theory}, and \textit{Sheaves and Geometry} \\

Teaching assistant, Johannes Gutenberg-Universität Mainz, 2019-2022 \\
Exercise classes and grading for \textit{Riemannsche Flächen}, \textit{Grundlagen der Numerik}, \textit{Zahlentheorie}, and \textit{Mathematik für Physiker} \medskip \par


~\par

\subsection*{Talks}

\subsubsection*{Conference talks}

\textit{Symmetric products and puncturing Campana-special varieties} \\
Diophantine and Rationality Problems in Sofia. (11th March 2025)

\textit{Kobayashi--Ochiai’s finiteness theorem for Campana pairs of general type} \\
DIAMANT Symposium Spring 2024 in Utrecht. (11th April 2024)

\subsubsection*{Seminar and colloquium talks}

\textit{The stable reduction theorem for curves} \\
Algebraic Geometry Seminar in Nijmegen. (19th November 2024) \\

\textit{Points of low degree on smooth projective curves} \\
Algebraic Geometry Seminar in Nijmegen. (22nd October 2024) \\

\textit{Parshin's proof of Bombieri--Lang for subvarieties of abelian varieties} \\
Seminar on Lang's Conjectures 2023 in Nijmegen. (1st December 2023) \\

\textit{The finiteness theorem of Kobayashi--Ochiai} \\
Seminar on Lang's Conjectures 2023 in Nijmegen. (15th September 2023)


\end{document}